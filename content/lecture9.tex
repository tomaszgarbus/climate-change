\section{Lecture 9: The Anthropocene -- a new geological epoch?}

\subsection{Anthropocene}

A book from 1874 by George P. Marsh citing Antonio Stoppani (1873) using the
term \textit{Anthropozoic Era}.

Eras are much longer than epochs.

Paul Crutzen, 2002: The Anthropocene started in the latter part of the
18th century, around the time of design of steam engine. More specifically:
1784.

Bryan Lovell, 2010, \textit{Challenged by carbon} 

\textbf{International Commission on Stratigraphy} $\rightarrow$
\textbf{Subcommission on Quaternary Stratigraphy} $\rightarrow$
\textbf{Anthropocene Working Group}

"to be accepted as formal geological time term the Anthropocene needs to be
(a) scientifically justified and (b) useful as a formal term to the scientific
community"

There are now 25k scientific articles using the term Anthropocene, only half of
them in geology.

\subsection{What will be preserved in the geological record of the
Anthropocene?}

\begin{itemize}
	\item an order-of-magnitude increase in erosion and sediment transport
	 associated with urbanization and agriculture;
	\item marked and abrupt anthropogenic perturbations of the cycles of
	 elements such as carbon, nitrogen, phosphorus and various metals
	 together with new chemical compounds;
	\item environmental changes generated by these perturbations, including;
	 global warming, sea-level rise, ocean acidification and spreading
	 oceanic "dead zones";
	\item rapid changes in the biosphere both on land and in the sea, as a
	 result of habitat loss, predation, explosion of domestic animal
	 populations and species invasions;
	\item the proliferation and global dispersion of many new "minerals"
	 and "rocks" including concrete, fly ash and plastic, and the myriad
	 "technofossils" produced from these and other materials;
\end{itemize}

A proposal to formalize the Anthropocene by the working group
\begin{itemize}
	\item formalized at the Epoch level (Phanerozoic Eon, Cenozoic Era,
	 Quaternary Period)
	\item it begins in 1950
	\item it would optimally be placed in the mid-20th century, coinciding
	 with the array of geological proxy signals preserved within recently
	 accumulated strata and resulting from the "Great Acceleration"
	\item Great Acceleration: population, global economy, global energy
	 consumption, atmospheric CO$_2$
	\item The sharpest and most globally synchronous of these signals,
	 that may be a primary marker, is made by the artificial radionuclides
	 spread worldwide by the thermonuclear bomb tests
\end{itemize}

\textbf{Suess effect}

\subsection{Crawford Lake}
\begin{itemize}
	\item everything around it is limestone
	\item water can be quickly saturated with calcium carbonate
	\item when it is warm, limestone precipitates to the bottom of the lake
	\item when it is cold, a regular clay sediments on the bottom
	\item we have annual layers of light (summer) layers and dark (winter)
	 layers
	\item there is a clear spike ("Golden Spike") corresponding to the
	 bomb tests
\end{itemize}

\subsection{Anthropocene}

The proposal for the Anthropocene Epoch was rejected last year, after 15 years
of debate.

\subsection{Challenging the rejection}

How does the case for the Anthropocene compare with the cases for previous
epochs?

\subsubsection{Eocene}

The start of a hypethermal defines the boundary between Paleocene and Eocene.
It was short lived and had a rapid onset. It was associated with two doublings
of the CO$_2$ in the atmosphere.
\begin{itemize}
	\item Anthropocene warming is happening 50-80 times faster than at the
	 onset of the hyperthermal.
	\item A full recovery from Ahtropocene warming will take as long as
	 from the hyperthermal (200k years).
	\item At its current rate. Anthropocene warming will reach $6\degree$
	 within 2-3 warmings.
\end{itemize}

\subsubsection{Oligocene}

The end of a major climatic shift from GH to icehouse conditions with global
cooling of 2.5$\degree$ in 500 ka.

\begin{itemize}
	\item Antropocene warming is happening 4000x faster than the cooling
	 during the Eocene-Oligocene transition
\end{itemize}

\subsubsection{Miocene}

A transient (400k years) global cooling event of at least $2\degree$, coupled
with a near doubling of the size of the Antarctic ice sheet.

\begin{itemize}
	\item Anthropocene warming is also (presently) associated with a
	 150ppm change of atmospheric CO$_2$ levels.
	\item A full recovery from Anthropocene warming will also take
	 hundreds of thousands of years.
\end{itemize}

\subsubsection{Holocene}

The start of the warm episode (interglacial) that began with the end of the
last glacial period.

Is the Holocene different than previous interglacials?

Is the Anthropocene different than the Holocene? From 2013 the temperatures
were no longer within the Holocene's temperature anomaly.
