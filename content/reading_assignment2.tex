\section{Geological methods for studying climate}

4 major archives of Earth's climatic history:
\begin{itemize}
	\item sediments
	\item ice
	\item corals
	\item trees
\end{itemize}

Sedimentary debris deposited by water is the major climate archive on Earth
for over 99\% of geological time.

\subsection{Sediments}
Sediment layers:

\begin{itemize}
	\item lake sediments
	\item interior sea sediments
	\item coastal margin sediments
	\item deep-ocean sediments
\end{itemize}

Preservation of older sedimentary records is hindered by two factors:
tectonic activity and erosion.

\textbf{Moraines} are long curving ridges made up of a jumbled mix of unsorted
debris carried by ice, ranging from large boulders to very fine clay.

\textbf{Loess} are sequences depositing silt-sized grains gathered by wind.

\subsection{Ocean sediments}
Ocean sediments are useful for researching last 150 Myr.

\subsection{Ice sheets}
Ice recovered from Antarctic ice sheet now dates back to 800000 years, while
Greenland's ice sheet just beyond 125000 years. Many small glaciers record only
the last 10000 years.

\subsection{Other climate archives}
Caves contain limestone deposits spanning several hundred thousand years.

Trees contain up to thousands of years of archives in annual layers.

Corals form annual bands of calcium carbonate (CaCO$_3$) or magnesium
carbonate (MgCO$_3$) that hold geochemical information about climate.
Individual corals may live for time span of up to hundreds of years.

Within the last few thousand years, people have also kept historical archives
of climate-related phenomena.

In last 100 to 200 years we also have instrumental records.

\subsection{Radiometric dating and correlation}

Scientists use \textbf{radiometric dating} to measure the decay of radioactive
isotopes\footnote{
	Isotopes are forms of a chemical element that have the same atomic
	number but differ in mass.
} in rocks. Dates are obtained on hard crystalline igneous rocks that once
were molten and then cooled to solid form.

In the second step, dates obtained from the igneous rocks provide constraints
on the ages of sedimentary rocks that occur in layers between the igneous rocks
and form the main archives of Earth's early climate history.

Radiometric dating is based on the radioactive decay of a \textbf{parent
isotope} to a \textbf{daughter isotope}. Parent is an unstable radioactive
isotope of one element and radioactive decay transforms it into the stable
isotope of another element (daughter).

The decay occurs at a constant rate which allows to use it as a clock.

Basalt is an igneous rock commonly used for datin. It cools quickly from molten
outpourings of lava. The event that starts the clock ticking is the cooling of
this material to the point where neither the parent nor the daughter isotope
can migrate in or out of the molten mass. At this point, the rock forms a
closed system, one in which the only changes occurring are caused by internal
radioactive decay.

Factors that complicate radiometric dating:
\begin{itemize}
	\item Initial abundance of daughter isotope is rarely 0
	\item System is not fully closed
\end{itemize}

The age of sediment layers can be obtained from the nearby igneous rocks.

\textbf{Fossil correlation} method relies on the fact that a unique and
unrepeated sequence of organisms has appeared and disappeared through Earth's
entire history and left fossilized remains

\subsection{Radiocarbon}

Radiocarbon dating is widely used to date lake sediments and other kinds of
carbon-bearing archives. Neutrons that constantly stream into Earth's
atmosphere from space convert $^{14}$N (nitrogen gas) to $^{14}C$ (an
unstable isotope of carbon). Vegetable and animal life forms on Earth extract
this carbon from the atmosphere to build both their hard shells and soft
tissue, and a small part of the carbon they extract is the radioactive $^{14}$C
isotope. The death of plant or animal closes off carbon exchanges with the
atmosphere and starts the decay clock. The $^{14}$C parent decays to the
$^{14}$N daughter and escapes to atmosphere as gas. The amount of $^{14}$C
that has been lost by the time a sample is analyzed can be determined by
measuring a different isotope of carbon that is stable.

Half-life of $^{14}$C carbon is 5780 years. Radiocarbon dating is most useful
over five or six half-lives.

\subsection{Counting annual layers}
Some climate repositories contain annual layers:

\begin{itemize}
	\item \textbf{mountain glaciers and ice sheets}: alternations between
	darker layers that contain dust blown from continental source regions
	during the dry cold windy season, and lighter layers marking the
	warmer part of the year with little or no dust.
	\item \textbf{varves} are annual couplets in some lakes, in particular
	deeper parts of lakes containing little or no life-sustaining
	oxygen. Lack of bottom-dwelling organisms helps preserve the thin
	annual layers.
	\item \textbf{tree rings} in regions with distinct seasons are
	alternations between lighter wood issue (cellulose) grown in spring
	and thin dark layers from autumn and winter
	\item \textbf{coral bands} record seasonal changes in the texture
	of the calcite (CaCO$_3$) incorporated in corals' skeletons.
\end{itemize}

\subsection{Climatic Resolution}

Factors which control the ability to resolve infromation from climatic
archives:
\begin{itemize}
	\item amount of disturbance of the sedimentary record by various
	processes soon after deposition
	\item the rate at which the record is buried beneath additional
	sediments and therebey protected from further disturbance
\end{itemize}
