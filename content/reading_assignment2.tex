\section{Geological methods for studying climate}

4 major archives of Earth's climatic history:
\begin{itemize}
	\item sediments
	\item ice
	\item corals
	\item trees
\end{itemize}

Sedimentary debris deposited by water is the major climate archive on Earth
for over 99\% of geological time.

\subsection{Sediments}
Sediment layers:

\begin{itemize}
	\item lake sediments
	\item interior sea sediments
	\item coastal margin sediments
	\item deep-ocean sediments
\end{itemize}

Preservation of older sedimentary records is hindered by two factors:
tectonic activity and erosion.

\textbf{Moraines} are long curving ridges made up of a jumbled mix of unsorted
debris carried by ice, ranging from large boulders to very fine clay.

\textbf{Loess} are sequences depositing silt-sized grains gathered by wind.

\subsection{Ocean sediments}
Ocean sediments are useful for researching last 150 Myr.

\subsection{Ice sheets}
Ice recovered from Antarctic ice sheet now dates back to 800000 years, while
Greenland's ice sheet just beyond 125000 years. Many small glaciers record only
the last 10000 years.

\subsection{Other climate archives}
Caves contain limestone deposits spanning several hundred thousand years.

Trees contain up to thousands of years of archives in annual layers.

Corals form annual bands of calcium carbonate (CaCO$_3$) or magnesium
carbonate (MgCO$_3$) that hold geochemical information about climate.
Individual corals may live for time span of up to hundreds of years.

Within the last few thousand years, people have also kept historical archives
of climate-related phenomena.

In last 100 to 200 years we also have instrumental records.

\subsection{Radiometric dating and correlation}

Scientists use \textbf{radiometric dating} to measure the decay of radioactive
isotopes\footnote{
	Isotopes are forms of a chemical element that have the same atomic
	number but differ in mass.
} in rocks. Dates are obtained on hard crystalline igneous rocks that once
were molten and then cooled to solid form.

In the second step, dates obtained from the igneous rocks provide constraints
on the ages of sedimentary rocks that occur in layers between the igneous rocks
and form the main archives of Earth's early climate history.

Radiometric dating is based on the radioactive decay of a \textbf{parent
isotope} to a \textbf{daughter isotope}. Parent is an unstable radioactive
isotope of one element and radioactive decay transforms it into the stable
isotope of another element (daughter).

The decay occurs at a constant rate which allows to use it as a clock.

Basalt is an igneous rock commonly used for datin. It cools quickly from molten
outpourings of lava. The event that starts the clock ticking is the cooling of
this material to the point where neither the parent nor the daughter isotope
can migrate in or out of the molten mass. At this point, the rock forms a
closed system, one in which the only changes occurring are caused by internal
radioactive decay.

Factors that complicate radiometric dating:
\begin{itemize}
	\item Initial abundance of daughter isotope is rarely 0
	\item System is not fully closed
\end{itemize}

The age of sediment layers can be obtained from the nearby igneous rocks.

\textbf{Fossil correlation} method relies on the fact that a unique and
unrepeated sequence of organisms has appeared and disappeared through Earth's
entire history and left fossilized remains

\subsection{Radiocarbon}

Radiocarbon dating is widely used to date lake sediments and other kinds of
carbon-bearing archives. Neutrons that constantly stream into Earth's
atmosphere from space convert $^{14}$N (nitrogen gas) to $^{14}C$ (an
unstable isotope of carbon). Vegetable and animal life forms on Earth extract
this carbon from the atmosphere to build both their hard shells and soft
tissue, and a small part of the carbon they extract is the radioactive $^{14}$C
isotope. The death of plant or animal closes off carbon exchanges with the
atmosphere and starts the decay clock. The $^{14}$C parent decays to the
$^{14}$N daughter and escapes to atmosphere as gas. The amount of $^{14}$C
that has been lost by the time a sample is analyzed can be determined by
measuring a different isotope of carbon that is stable.

Half-life of $^{14}$C carbon is 5780 years. Radiocarbon dating is most useful
over five or six half-lives.

\subsection{Counting annual layers}
Some climate repositories contain annual layers:

\begin{itemize}
	\item \textbf{mountain glaciers and ice sheets}: alternations between
	darker layers that contain dust blown from continental source regions
	during the dry cold windy season, and lighter layers marking the
	warmer part of the year with little or no dust.
	\item \textbf{varves} are annual couplets in some lakes, in particular
	deeper parts of lakes containing little or no life-sustaining
	oxygen. Lack of bottom-dwelling organisms helps preserve the thin
	annual layers.
	\item \textbf{tree rings} in regions with distinct seasons are
	alternations between lighter wood issue (cellulose) grown in spring
	and thin dark layers from autumn and winter
	\item \textbf{coral bands} record seasonal changes in the texture
	of the calcite (CaCO$_3$) incorporated in corals' skeletons.
\end{itemize}

\subsection{Climatic Resolution}

Factors which control the ability to resolve infromation from climatic
archives:
\begin{itemize}
	\item amount of disturbance of the sedimentary record by various
	processes soon after deposition
	\item the rate at which the record is buried beneath additional
	sediments and therebey protected from further disturbance
\end{itemize}

\subsection{Climate proxies}

Scientists must first determine the mechanism by which climate signals are
recorded by the proxy indicators in order to use them to decipher climate
changes.

Two climate proxies most commonly used:
\begin{itemize}
	\item Biotic proxies, based on canges in composition of plant and
	animal groups
	\item Geological-geochemical proxies, measurements of mass movements
	of materials through the climate system, either as discrete (physical)
	particles or in dissolved (chemical) form.
\end{itemize}

Abundance of \textbf{pollen} helps reconstruct the climate on land for younger
time intervals.

Types of shelled remains of plankton:

\begin{itemize}
	\item CaCO$_3$: foraminifera
	\item CaCO$_3$: coccoliths
	\item SiO$_2$: diatoms
	\item SiO$_2$: radiolaria
\end{itemize}

Sediments rich in CaCO$_3$ fossils occur in open-ocean waters at depths above
3500-4000 meters. Below that level, corrosive bottom waters dissolve calcite
shells. SiO$_2$-shelled diatoms inhabit deltas and other coastal areas and
extract silica from river water flowing off the land.

\subsection{Geological and geochemical data}

Sediment is eroded from the land and deposited in ocean basins in two forms:
\begin{itemize}
	\item \textbf{physical weathering}, the process by which water, wind
	and ice physically detach pieces of bedrock and reduce them to smaller
	fragments. Examples include ice-rafted debris (sand and gravel eroded
	by ice sheets and delivered by icebergs that melt in ocean), eolian
	sediments (silts and clays lifted from the continents and blown to
	ocean by winds).
	\item \textbf{chemical weathering} and subsequent transport of
	dissolved ions to the oceans in rivers. It occurs mainly in two ways:
	hydrolysis and dissolution.
\end{itemize}

\subsection{Key terms}

\textbf{Moraines}: long curving ridges created by retreating ice sheets.

\textbf{Loess}: sequences of silt\footnote{pl: muł} deposited by wind as thick
layers.

\textbf{Historical archives}: human-made records of climate-related phenomena,
available from last few thousand years.

\textbf{Instrumental records}: are available from last 100 to 200 years since
the emergence of the first thermometers in the eighteenth century.

\textbf{Radiometric dating}: Radiometric dating is based on the radioactive
decay of a parent isotope to a daughter isotope. Time elapsed is measured by
comparing the abundance of these two and combining this data with the half-life
of parent.

\textbf{Parent isotope}: An unstable radioactive isotope of one element which
transforms through radioactive decay to another.

\textbf{Daughter isotope}: The isotope resulting from the process of
radioactive decay.

\textbf{Closed system}: A system in which the changes are not driven by
external factors but the content of the system itself, fx internal
radioactive decay.

\textbf{Half-life}: One half-life is the time needed for half the parent that
was present previously to decay to the daughter isotope.

\textbf{Radiocarbon dating}: Cosmic rays generate neutrons as they travel
through the atmosphere which can strike $^{14}$N atoms and turn them into
$^{14}$C, also called radiocarbon, which has a half time of 5730 years.
Radiocarbon atoms are ingested by animals and plants through the diet. Once
they die, they stop exchanging carbon with their surroundings, so the content
of radiocarbon in their organisms decreases. Since it decreases at a known rate
we can estimate the organisms' age. The daughter isotope of $^{14}$C is again
nitrogen $^{14}$C. Radiocarbon dating was invented by Willard Libby.

\textbf{Varves}: annual couplets, sediments in some lakes. They result from
seasonal alternatoins between deposition of light-hued mineral-rich debris
and darker sediment rich in organic material.

\textbf{Tree rings}: alternations between thick layers of lighter wood tissue
(cellulose) formed by rapid growth in spring, and thin dark layers marking
cessation of growth in autumn and winter.

\textbf{Coral bands}: recorded seasonal changes in the texture of the calcite
(CaCO$_3$) in corals' skeletons. The lighter parts are laid down during
intervals of fast growth and the darker layers when growth slows.

\textbf{Climate proxies}: indicators of past climate, indirect signals of
past climate.

\textbf{Biotic proxies}: climate proxies based on changes in composition of
plant and animal groups.

\textbf{Geological-geochemical proxies}: measurements of mass movements of
materials through the climate system, either as discrete (physical) particles
or in dissolved (chemical) forms.

\textbf{Macrofossils}: larger vegetation remains that cannot have been carried
far from their points of origin, such as cones, seeds and leaves.

\textbf{Plankton}: diverese collection of organisms that drift in water or air
but are unable to actively propel themselves against current or wind.

\textbf{Planktic foraminifera}: globular sand-sized animals that inhabit upper
layers of the ocean.

\textbf{Coccoliths}: individual plates or scales of coccolithophores made of
calcium carbonate (CaCO$_3$).

\textbf{Diatoms}: hard-shelled plankton, silt-sized plant plankton shaped
like pillboxes or needles. Secrete shells of opaline silica
(SiO$_2$ $\cdot$ H$_2$O).

\textbf{Radiolaria}: sand-sized animals with ornate shells often resembling
pre-modern (Prussian) military helmets. Create shells of opaline silica
(SiO$_2$ $\cdot$ H$_2$O).

\textbf{Burial fluxes}: measures of the mass of sediment deposited per unit
area per unit time. Useful for mapping the changes in the patterns of
deposition of ocean sediments spanning the last 170 million years.

\textbf{Physical weathering}: the process by which water, wind and ice
physically detach pieces of bedrock and reduce them to smaller fragments.

\textbf{Ice-rafted debris}: a type of sediment created by physical weathering.
It results from eroded ice sheets and is delivered by icebergs that melt in
ocean waters. It consists of sand and gravel.

\textbf{Eolian sediments}: silts and clays lifted from the continents and blown
to the ocean by winds.

\textbf{Fluvial sediments}: refers to the process of creating deposits by
carrying sediments by rivers and streams.

\textbf{Chemical weathering}: a second major way of removing sediments from
the land (the other one is physical weathering). Chemical weathering happens
through changing the chemical composition of the rock. Subsequently, the
dissolved ions are transported to the oceans in rivers.

\textbf{Dissolution}: one of two types of chemical weathering, in which
carbonate rocks (such as limestone, made of CaCO$_3$) and evaporite rocks
(such as rock salt, made of NaCl) are dissolved in water.

\textbf{Hydrolysis}: one of two types of chemical weathering, in which the
weathering process adds water to the minerals derived from continental rocks
made of silicates, such as basalts and granites.

\textbf{Benthic foraminifera}: sand-sized animals that live on the seafloor and
form calcite (CaCO$_3$) shells from Ca$^{+2}$ and CO$_{3}^{-2}$ ions in deep
waters.

\textbf{Physical climate models}: numerical (computer) models used by climate
scientists which emphasize the physical operation of the climate system,
particularly the circulation of the atmosphere and ocean but also interactions
with global vegetation (biology) and with atmospheric trace gases (chemistry).

\textbf{Geochemical models}: models that track the movement of distinctive
chemical tracers throughthe climate system.

\textbf{Control case}: simulation of the modern climate. Models must be
capable of simulating modern climate reasonably well in order to be trusted for
exploring past climates.

\textbf{Boundary conditions}: the features that are altered to test hypotheses
of climate change, such as height of mountains, presence of ice sheets, level
of CO$_2$ in the atmosphere.

\textbf{Climate simulation}: the process of runing a climate model during an
experiment.

\textbf{Climate data output}: the climatic data produced in a simulation.

\textbf{Aerosols}: airborne particles, such as volcanic ash and dust.

\textbf{Atmospheric general circulation models (A-GCMs)}: three dimensional
climate models  that incorporate many key features of the real world: the
spatial distribution of land, water and ice, the elevation of mountains and
ice sheets, the mount of vertical distribution of greenhouse gases in the
atmosphere, the seasonal variations in solar radiation.

\textbf{Grid boxes}: boundary conditions for A-GCM experiments are specified
for hundreds of model grid boxes.

\textbf{Sensitivity test}: An approach to A-GCM experiments, where one
boundary condition at a time is altered in relation to the present
configuration. When the output of such an experiment is compared to the
output from the modern control case, the differences in climate between the
two runs isolate and reveal the unique impact caused by the change in that one
boundary condition.

\textbf{Reconstruction}: An experiment with climate model where all known
boundary conditions are changed at the same time in order to try to simulate
the full state of the climate system at some time in the past.

\textbf{Ocean general circulation models (O-GCMs)}: are usually simpler than
A-GCMs. One reason is that climate researchers know much less about the
modern circulation of the oceans, especially critical processes such as the
brief but intense episodes when deep water forms at high latitudes.

\textbf{Geochemical tracers}: chemical materials that are tracked by
geochemical circulation models.

\textbf{Reservoirs}: environments containing the chemicals, such as the
atmosphere, ocean, ice, vegetation and sediments.

\textbf{Mass balance models}: models that trace the movements of chemicals
between reservoirs.

\textbf{Residence time}: the average time it takes for a geochemical tracer to
pass through a reservoir. For a reservoir at a steady state (equal pace of
output and input flux), this time is equal to:
$\text{residence time} = \frac{\text{reservoir size}}{\text{Flux rate in
(or out)}}$

\subsection{Review questions}

\subsubsection{Why does the importance of different climate archives change
for different time scales?}
Different climate archives vary in precision and availability back in time:
\begin{itemize}
	\item \textbf{Instrumental}: 10$^2$-0 years back
	\item \textbf{Historical}: 10$^3$-0 years back
	\item \textbf{Tree rings}: 10$^4$-0 years back
	\item \textbf{Ice cores}: 10$^6$-0 years back
	\item \textbf{Lake sediments}: up to $10^9$ years back
	\item \textbf{Coral reefs}: 10$^6$-0 years back
	\item \textbf{Ocean sediments}: up to $10^9$ years back
	\item \textbf{Continental coastal sediments}: up to $10^9$ years back
\end{itemize}

\subsubsection{Why are ocean sediments and ice cores especially important
archives of climate?}
Deep ocean is generally undisturbed with relatively continuous deposition.
It usually yields climate records of higher quality than records from land,
where water, ice and wind repeatedly erode deposits.

Ice cores retrieve climate records extending back thousands of years in small
mountain glaciers to as much as hundreds of thousands of years in
continent-sized ice sheets.

\subsubsection{How does the method of dating climate records vary with the
type of archive?}
For vegetation and organisms, radiocarbon dating is used. For many kinds of
rocks. radiometric dating and correlation. Counting annual layers works for
some sediments in water, tree rings, or coral bands. Orbital cycles can be
used to date low-latitude monsoons and the growth-decay cycle of
high-latitude ice sheets.

\subsubsection{How does the resolution from sedimentary archives vary with
depositional environment?}
In the oceans, four groups of shell-forming animal and plankton are used for
reconstructions, two of which made of calcite and two of opaline silica.

In lakes, annual sediment varves are used.

On bedrock, annual ice layers can be analysed.

\subsubsection{Which two major groups of organisms are most important to
climate reconstructions over the past several million years?}

Plankton and pollen (vegetation).

\subsubsection{Describe how the products derived from physical and chemical
weathering provide different kinds of information about the climate system.}

Physically weathered sediments reveal the climate of the source regions, for
example grains of quartz and other minerals from ice sheets indicate cold
climates.

Chemically weathered sediments record changes in the global volume of ice and
in local ocean temperatures.

\subsubsection{Describe the two ways the performance of climate models is
evaluated.}
\begin{itemize}
	\item control case -- simulation of modern climate
	\item testing climate data output against independent geologic data
	that played no part in the experimental design
\end{itemize}

\subsubsection{Why aren't models of the atmosphere and ocean allowed to
interact continuously?}
Air and water responde to climate changes at different rates that impose
different computational demands. Ocean models can ignore interactions that
occur on a daily cycle because these short-term changes have negligible effects
on most ocean circulation. As a result, O-GCMs generally only need to calculate
changes over timesteps separated by a month or more. By contrast, daily changes
are critical to models of the fast-responding atmosphere.

\subsubsection{Describe two features that make the ocean useful in geochemical
mass balance models.}
\begin{itemize}
	\item isotopic composition of oxygen in the H$_2$O molecules that are
	deposited in ice sheets differs from the average composition of the
	molecules left in the ocean.
	\item terrestial carbon (vegetation) is enriched in one isotope of
	carbon compared to the average in the ocean
\end{itemize}
