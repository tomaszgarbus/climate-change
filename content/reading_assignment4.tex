\section{Reading assignment 4: Plate Tectonics and Long-Term Climate}

In 1914 the German meteorologist Alfred Wegener proposed that continents move
slowly across Earth's surface. He was almost right: in fact \textit{all} of
Earth's surface slowly moves.

The best rocks to use as ancient compasses are basalts, which are rich in
highly magnetic iron. Basalts form the floors of ocean basins and are also
found on land in actively tectonic regions.

As the molten material cools, its iron-rich components align with Earth's
magnetic field like a compass. After lava turns into basaltic rock (when its
temperature drops below $1200 \degree$, continued cooling to temperatures near
$600 \degree$ allows the "fossilized" magnetic compases to become fixed in
position in the rock.

Also locked in the basalts are radioactive materials such as potassium (K)
which help date the basalt layers.

The age of the ocean crust steadily increases with distance from the ridges
(at divergent margins).

Since 450 Myr ago, major (continent-sized) ice sheets have existed on Earth
during three icehouse earas:
\begin{itemize}
	\item a brief interval centered near 445 Myr ago
	\item much longer interval from 325 to 240 Myr ago
	\item current icehouse era of the last 35 Myr
\end{itemize}

Most CO$_2$ is expelled to the atmosphere by volcanic activity at two kinds of
locations:

\begin{itemize}
	\item margins of converging plates, where parts of the subducting
	litosphere melt and form molten magmas that rise to the surface in
	mountain belt and island arc volcanoes, delivering CO$_2$ and other
	gases from Earth's interior
	\item where hot magma carrying CO$_2$ erupts directly into ocean water
\end{itemize}

\subsection{Key terms}

\textbf{continental crust}: the layer of igneous, metamorphic and sedimentary
rocks that forms the geological continents. It is 30-70 kilometers thick, has
an average composition like that of granite and is low in density
($2.7 \frac{g}{cm^3}$).

\textbf{ocean crust}: uppermost layer of the oceanic portion tectonic plates.
Average ocean floor is 4000 meters below sea level. Ocean crust is 5-10km
thick, has average composition like that of basalt and is higher in density
($3.2 \frac{g}{cm^3}$).

\textbf{mantle}: the layer of Earth that lies below both continental
and oceanic rust. It extends halfway towards the Earth's interior (2890km out
of 6370km). It is richer in heavy elements like iron (Fe) and magnesium (Mg).
It has dencity of $3.6 \frac{g}{cm^3}$).

\textbf{litosphere}: rock layer below Earth's surface. It is 100km thick and
generally behaves like hard, rigid substance. It enxompasses the crustal
layers and the upper part of the underlying mantle.

\textbf{asthenosphere}: the layer below litosphere, lying entirely within the
upper section of Earth's mantle at depths of 100 to 350 kilometers. It is
partly molten but mostly solid. It behaves like a soft, viscous fluid over long
intervals of time, and flows more easily.

\textbf{tectonic plates}: a division of the litosphere. These plates move at
rates ranging from <1 to 10 cm per year and average about the same rate as
growth of a fingernail.

\textbf{divergent margins}: one of 3 types of edges (margins) of tectonic
plates, occurring when plates move apart, for example in the middle of
Atlantic Ocean. This motion allows new ocean crust to be created. Plates
diverging at ocean ridges carry not just the near-surface layer of ocean crust
but also a much thicker layer of mantle lying underneath.

\textbf{convergent margins}: another type of edges (margins) of tectonic plates
occurring when plates come together.

\textbf{subduction}: a process occuring at convergent margins. The litosphere
(ocean crust and upper mantle) plunges deep into Earth's interio and ocean
trenches. For example, narrow mountain chains such as the Andes from on the
adjacent continnts because of the squeezing (compressing) forces produced when
the two plates move together.

\textbf{continental collision}: a less common type of converging plates. It
creates massive high-elevation regions such as the Tibetan Plateau.

\textbf{transform fault margins}: the last type of tectonic plates' edges, when
plates slide past each other. Sliding of plates at transform faults involves
the litosphere, both the upper continental crust and the underlying upper
mantle.

\textbf{magnetic field}: the Earth has a magnetic field that determines the
alignment of compass needles. Magnetic north is located a few degrees of
latitude away from geographic North Pole.

\textbf{paleomagnestism}: the study of prehistoric Earth's magnetic fields
recorded in rocks, sediments and archeological materials.

\textbf{magnetic lineations}: stripes of normal and reversed polarity records
growing symmetrically out of ocean ridges.

\textbf{seafloor spreading}: the process of divergence of litospheric plates
at ocean crust.

\textbf{polar position hypothesis} makes two predictions:
\begin{itemize}
	\item ice sheets should appear on continents that were located at
	polar or near-polar latitudes
	\item no ice should appear at times when continents were located
	outside of polar regions
\end{itemize}
In other words, it claims that land presence in polar region is both the
sufficient and necessary condition for glacier formation.

However, this hypothesis can be confirmed only one way: indeed polar
positioning of land is favorable to formation of glaciers, but it seems
not sufficient.

\textbf{Gondwana}: a large southern supercontinent consisting of modern
Africa, Arabia, Antarctica, Australia, South America and India. It was located
on the opposite side of the globe from North America, but it had begun a long
trip that would carry it across the South Pole.

\textbf{Pangaea}: the giant supercontinent, meaning "All Earth"

\textbf{evaporite}: deposits that precipitated out of water in lakes and
coatal margin basins with limited connections to the ocean. Evaporite salts
form only in arid regions where evaporation far exceeds precipitation.
More evaporite salt was deposited during the time of Pangaea than at any time
in the last several hundred million years.

\textbf{red beds}: sandy or silty sedimentary rocks stained various shades of
red by oxidation of iron minerals. Red-colored soils accumulate today in
regions where the contrast in seasonal moisture is strong.

\textbf{spreading rate hypothesis}: a hypothesis proposed in 1983 that the
climate changes during the last several hundred million years were driven
mainly by changes in the rate of CO$_2$ input to the atmosphere and ocean by
plate tectonic processes.

\textbf{hot spots}: volcanic hotspots are locales where thin plumes of molten
material rise from deep within the interior and reach the surface.

\textbf{uplift weathering hypothesis}: a proposition that chemical weathering
is an active driver of climate change, rather than just a passive negative
feedback that moderates climate. The hypothesis focuses on evidence that
exposure of fragmented and unweathered rock is a key factor in the intensity
of chemical weathering. It then links this evidence to the fact that freshly
fragmented rock is exposed mainly in regions of tectonic uplift.

\textbf{mass wasting}: erosional processes producing rock slides and falls,
flows of water-saturated debris, and a host of other processes that dislodge
everything from huge slabs of rock to loose boulders, pebbles and soil.

Uplift $\rightarrow$ (steep slopes, mass wasting, mountain glaciers, slope
precipitation) $\rightarrow$ increased rock fragmentation $\rightarrow$
increased weathering and CO$_2$ removal $\rightarrow$ global cooling

\subsection{Review questions}

\subsubsection{Does each litospheric plate correspond to an individual
continent or ocean basin?}
Most tectonic plates correspond to a combination of a continent and a part of
ocean basin.

\subsubsection{What kind of physical behaviour in Earth's deeper layers allows
the plates to move?}
Molten fluids circulating in Earth's liquid iron core create a magnetic field
analogous to that of a bar magnet.

\subsubsection{Explain how paleomagnetism tells us about past latitudes of
continents.}
The orientations of magnetic compases frozen in basalt layers deposited on
continents in relation to the magnetic poles are used. In molten lavas that
cool at high latitudes, the internal magnetic comasses point in a nearly
vertical direction because Earth's magnetic field has that orientation at high
latitudes. In contrast, lavas that cool near the equator have internal
compasses oriented closer to horizontal, nearly parallel to Earth's surface.

\subsubsection{Explain how paleomagnetism tells us about rates of spreading of
ocean ridges.}
Past changes in the magnetic field (inversed polarity) are recorded in fossil
magnetic compasses in well-dated basaltic rocks from many regions. These
changes are also recorded in magnetic lineations -- stripes of normal and
reversed polarity recorded in ocean crust -- growing symmetrically out of ocean
ridges.

\subsubsection{Do glaciations \textit{always} occur when continents are located
in polar positions?}
No, there were greenhouse eras (no glaciations) with continents in polar
regions.

\subsubsection{What are the major characteristics of the climate of Pangaea?}
No ice sheets existed even though its norther and southern limits ay within
Arctic and Antarctic circles. This suggests that Pangaea's climate was
somewhat warmer than today's climate. This is also supported by fossil evidence
of vegetation. Several kinds of palm-like vegetation that would have been
killed by hard freezes existed on Pangaea to latitudes as high as $40 \degree$.
Because the moderating effcts of ocean moisture failed to reach much of
Pangaea's interior, the continent was left vulnerable to seasonal extremes of
solar heating in summer and cooling during winter.

\subsubsection{What is the central concept behind the BLAG (spreading rate)
hypothesis?}
That changes in the average rate of seafloor spreading over millions of years
have controlled the rate of delivery of CO$_2$ to the atmosphere from the
large subsurface rock reservoir of carbon and that the resulting changes in
atmospheric CO$_2$ concentrations have had an impact on Earth's climate.

\subsubsection{What role does chemical weathering play in the BLAG hypothesis?}
It creates a negative feedback. Warmer climate leads to more precipitation,
thus more vegetation, thus higher removal of CO$_2$ from the atmosphere.

\subsubsection{Write a chemical reaction showing how weathering removes
CO$_2$ from the atmosphere.}
$\underset{\text{Silicate rock}}{\text{CaSiO}_2} +
\underset{\text{Atmosphere}}{\text{CO}_2}
\rightarrow
\underset{\text{Plankton}}{\text{CaCO}_3}
+
\underset{\text{Plankton}}{\text{SiO}_2}$

\subsubsection{How soon after deposition does freshly fragmented debris undergo
most chemical weathering?}

\subsubsection{Why is chemical weathering faster in the eastern Andes than in
the Amazon lowlands?}
The minerals in Amazon lowlands have long been "used up" in the weathering
process, while the physical impacts of active uplift in the Andes (steep
slopes, earthquakes, mass wasting, heavy precipitation, and glacial erosion)
combine to generate a continual supply of fresh, finely ground rock debris for
weathering.

\subsubsection{How could chemical weathering be both the driver and the
thermostat of Earth's climate?}
The effects of the uplift weathering processes are probably opposed by the
chemical weathering thermostat.

\subsubsection{Fast subduction in the modern Pacific Ocean carries down
sediments with low amounts of CaCO$_3$ while almost no subduction occurs in
the Atlantic Ocean, with its carbonate-rich sediments. What would happen if
subduction suddenly began in the Atlantic and replaced an equal amount of
subduction in the Pacific?}
