\section{Reading assignment: Chapter 19: Causes of Warming over the last 125
years}

\subsection{Natural causes of recent warming}
Changes over tectonic time scales are clearly irrelevant to the changes of the
last 125 years. Average rate of cooling during transitions between greenhouse
and icehouse conditions is $0.00001\degree$ per century).

Orbital forcing works at the rate of $0.00016 \degree$ per century or less.

\subsubsection{Solar forcing}

The amount of radiation arriving from the Sun varies in 11-year cycles changing
by a little over 1 W/m$^2$, equivalent to about $0.1\%$ of the global average.

\subsection{CO$_2$}

Bubbles of ancient air trapped in ice and direct measurements of air by the
geochemist Charles Keeling begun in 1958 show an accelerating rise in the
CO$_2$ concentration during the last two centuries. By 2010 the CO$_2$
concentrations had risen to 390 ppm, well above the 180-300ppm range of natural
(glacial-interglacial) variations.

On an annual average, cold high-latitude ocean water is a CO$_2$ sink and warm
low-latitude ocean is a CO$_2$ source.
One of the reasons is that CO$_2$ gas is more easily dissolved in cold water
than in warm water.

Greenhouse experiments show that most plants obtain carbon more effectively
from CO$_2$-rich atmosphere and grow faster as a result.

Ice core and instrumental measurements show that atmospheric CO$_2$ levels have
risen by almost $40\%$ in the last 150 years.

\subsection{Methane (CH$_4$)}

Since the 1800s, the methane concentration has risen to over 1750 ppb, well
above the natural range of 350-700ppb.

Methane comes from sources rich in organic carbon but lacking in oxygen, such
as swampy bogs with decaying plants, guts of cattle, animal and human waste,
burning of grassy vegetation.

\subsection{Increases in Chlorofluorocarbons (CFCs)}

These compounds include elements of chlorine (Cl), fluorine (Fl) and
bromine (Br).

They have for decades been produced for use in refrigerators and air
conditioner coolants, cheical solvents, fire retardants and foam insulation in
buildings.

They stay in the atmosphere for around 100 years.

\subsection{Ozone}

Ozone originates from both natural and human processes such as biomass burning
and oil production in refineries.

At high concentrations ozone is toxic to plants and human eyes and lungs.

In the lowermost atmosphere, ozone increased due to human activities, causing
periodic smog alerts in many large cities.

\subsection{Sulfate aerosols}

Large plumes of sulfate aerosols have a cooling effect on the climate by
reflecting some of the solar radiation back to space.
The second effect is less understood and that is particles of water vapour
condensing around aerosole particles and forming clouds.

\subsection{Land clearance}

Effects:
\begin{itemize}
	\item Increased albedo at high and middle latitudes
	\item At tropical and subtropical latitudes, reduced amounts of
	evapotranspiration\footnote{
		Evapotranspiration = evaportation + transpiration.
		Transpiration is the process of water movement through a
		plant and its evaporation from aerial parts, such as leaves,
		stems and flowers.
	}.
	\item Consequently, with reduced moisture, land surfaces dry out and
	bake in the sun.
	\item On a global average basis, the net effect of land clearance has
	been a small cooling of the planet.
\end{itemize}

\subsection{Climate feedbacks}
Positive feedbacks (evaluated at 2x CO$_2$)
\begin{itemize}
	\item Water vapor. According to current estimates it could cause
	additional $1.1 \degree$ to $1.5 \degree$ warming in addition to the
	$1.1 \degree$ from radiative forcing.
	\item Diminishing albedo due to retreat of snow and ice toward the
	poles. Estimated at about $0.6 \degree$ additional warming.
	\item Clouds. Difficult to estimate.
\end{itemize}

Negative feedbacks:
\begin{itemize}
	\item Aerosols seeding cloud nuclei.
\end{itemize}

Other effects slowing down climate change:
\begin{itemize}
	\item Ocean Thermal Intertia. Slows down the oceans' response by
	decades.
	\item Anthropogenic aerosols. Offsets the radiative forcing by up to
	25\%.
\end{itemize}

\subsection{Key terms}

\textbf{chlorofluorocarbons (CFCs)}: Compounds that include elements of
chlorine (Cl), fluorine (Fl) and bromine (Br).

\textbf{ozone}: O$_3$. It occurs naturally in the stratosphere, with the
largest concentrations at altitudes between 15 and 30 km. Incoming UV radiation
from the Sun liberates individual O atoms from oxygen (O$_2$) and produces
ozone.

\textbf{ozone hole}: Chlorine reacts with ozone and destroys it, forming
chlorine monoxide (ClO). The region over Antarctica in which stratospheric
ozone is much less abundant than elsewhere (due to cumulation of CFCs) is
called the ozone hole.

\textbf{brown clouds}: Carbon-rich aerosole hazes over Southeast Asia, mostly
originating from  small cook stoves in which people burn organic matter for
fuel, including cow dung.

\textbf{black carbon}: Carbon particles resulting from incomplete combusion,
e.g. soot. When they fall down and settle on bright surfaces over snow and
sea ice, they absorb solar radiation and reduce the albedo effect.

\textbf{global dimming}: A phenomenon in which the amount of solar radiation
reaching the ground slowly decreases due to sulfate aerosols, brown clouds,
contrails emitted by jets, and other emissions. Between 1950s and 1980s the
solar energy reaching the ground decreased by several percent.

\textbf{2x CO$_2$ sensitivity}: The global average temperature predicted by
a climate model (run to equilibrium or near-equilibrium) assuming the doubling
of CO$_2$ in the atmosphere from the preindustrial level of 280ppm.

\textbf{equivalent CO$_2$}: A unit standardizing different gases' global
warming potential in relation to CO$_2$. For instance, the multiplier for
methane is 25.

\textbf{radiative forcing}: Additional radiation hitting the Earth's surface
measured in W/m$^2$, caused by GHGs.

\textbf{enhanced greenhouse effect}: Greenhouse effect contributed since 1850,
that is excluding the natural greenhouse effect of 150W/m$^2$. Enhanced
greenhouse effect is 2.7W/m$^2$.

\subsection{Review questions}

\subsubsection{What human activities produce CO$_2$ and how have they changed
in the last 200 years?}

Late 1700s and most of the 1800s: clearing of forests for agriculture and
charcoal for furnaces in the early part of the Industrial Revolution.

After 1900: extraction of fossil fuels buried beneath Earth's surface: coal
at first, then oil and natural gas.

\subsubsection{Where does the CO$_2$ produced by humans go?}

\begin{itemize}
	\item Atmosphere: 55\%
	\item Biosphere: 15-20\%
	\item Shallow ocean: 25-30\%
\end{itemize}

\subsubsection{How high in the atmosphere do sulfate aerosols from smokestacks
reach?}

Within the lower several kilometers.

\subsubsection{Why do cholorfluorocarbons (CFCs) reach much higher in the
atmosphere than sulfate aerosols?}

Because CFCs stay in the atmosphere for around 100 years and SAs are removed by
rain after a few days.

\subsubsection{What are the strongest positive and negative feedbacks on
changes in Earth's temperature?}

Positive: water vapour. Negative: aerosols working as cloud nuclei.

\subsubsection{In a net sense, do feedbacks increase or decrease the direct
radiative effects of greenhouse gases on global temperature?}

On a short time scale they moderate this effect.

\subsubsection{What factors complicate attempts to estimate Earth's sensitivity
to CO$_2$ by directly comparing the observed twentieth-century warming to the
measured rise in greenhouse gases?}

Anthropogenic aerosols and ocean's thermal intertia, both of which slow down
or moderate the direct warming effect.

\subsubsection{Some climate skeptics point out that temperatures were warmer in
north polar regions 6000 years ago, and conclude that modern GHG concentrations
have not produced warmth that is unusual by natural standards. Evaluate the
relevance of this conclusion based on what you have learned from this book.}

None of the natural effects (orbital forcing, solar cycles, Milankovic cycles
etc.) work with such strong effect on such short time scale as we are observing
now.
