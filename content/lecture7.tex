\section{Lecture 7: A geological perspective on ongoing global warming}

\subsection{Atmospheric CO$_2$ concentration}

\begin{itemize}
	\item measurement data from ice cores (before 1959) and Mauna Loa
	(after 1959)
\end{itemize}

\subsection{Cumulative anthropogenic emission of carbon}
\begin{itemize}
	\item currently at over 400 PgC
	\item remaining fuels allow to hit 1000-2000 PgC
	\item PETM emissions: 3000-7000 PgC
\end{itemize}

\subsection{Atmospheric CO$_2$ concentration}
\begin{itemize}
	\item currently at around 350ppm
	\item remaining fossil fuels allow for 550-750ppm
\end{itemize}

\subsection{Eocene greenhouse}
\begin{itemize}
	\item sea level was 65 m higher
	\item temperature was $5-15\degree$ higher
	\item risk for hyperthermals
\end{itemize}

\subsection{Comparison of the PETM hyperthermal with ongoing global warming}
\begin{itemize}
	\item based on carbon emission rates, ongoing global warming is
	calculated to be an order-of-magnitude faster than at the start of the
	PETM hyperthermal
\end{itemize}

\subsection{"Eocene Park"}

\begin{itemize}
	\item if we burn remaining fossil fuels (within 50-100 years), Earth's
	climate will resemble the Eocene greenhouse frm 56 to 35 Ma
	\item in the Eocene, sea level was 65m higher and temperature was
	$10-15 \degree$ higher than today
	\item there were also hyperthermals such as the PETM
	\item ongoing global warming is an order-of-magnitude faster than
	global warming at the start of the PETM
	\item recovery from global warming is likely to take 170000 years
\end{itemize}
