\section{Lecture 7: A geological perspective on ongoing global warming}

\subsection{Atmospheric CO$_2$ concentration}

\begin{itemize}
	\item measurement data from ice cores (before 1959) and Mauna Loa
	(after 1959)
\end{itemize}

\subsection{Cumulative anthropogenic emission of carbon}
\begin{itemize}
	\item currently at over 400 PgC
	\item remaining fuels allow to hit 1000-2000 PgC
	\item PETM emissions: 3000-7000 PgC
	\item according to business-as-usual projections, we will enter the
	PETM levels in 2159 and go beyond them in 2278
	\item but this is not going to happen, because we'll run out of
	fossil fuels before that
\end{itemize}

\subsection{Atmospheric CO$_2$ concentration}
\begin{itemize}
	\item currently at around 350ppm
	\item remaining fossil fuels allow for 550-750ppm
	\item 2.13 PgM CO$_2$ corresponds to 1ppm change
	\item according to projections, we'll run out of fossil fuels between
	2070 and 2120 assuming business-as-usual
\end{itemize}

\subsection{Eocene greenhouse}
\begin{itemize}
	\item sea level was 65 m higher
	\item temperature was $5-15\degree$ higher
	\item risk for hyperthermals
	\item 35 Ma switch from greenhouse to icehouse
	\item burning out all fossil fuels puts us in the greenhouse conditions
\end{itemize}

\subsection{Comparison of the PETM hyperthermal with ongoing global warming}
\begin{itemize}
	\item based on carbon emission rates, ongoing global warming is
	calculated to be an order-of-magnitude faster than at the start of the
	PETM hyperthermal
	\item carbon accumulation rates in PETM: $0.3-1.5$ PgC/yr, currently it
	is $11$ PgC/yr (anthropogenic)
\end{itemize}

\subsection{"Eocene Park"}

\begin{itemize}
	\item if we burn remaining fossil fuels (within 50-100 years), Earth's
	climate will resemble the Eocene greenhouse frm 56 to 35 Ma
	\item in the Eocene, sea level was 65m higher and temperature was
	$10-15 \degree$ higher than today
	\item there were also hyperthermals such as the PETM
	\item ongoing global warming is an order-of-magnitude faster than
	global warming at the start of the PETM
	\item recovery from global warming is likely to take 170000 years
\end{itemize}

\subsection{Summary}
\begin{itemize}
	\item the biggest problem is not the amount of warming, it's the rate
	of warming
	\item Earth actually thrives in a warmer climate, provided that the
	warming happens over very long timescale and ecosystem has time to
	adapt (e.g. 60 Myr)
\end{itemize}
