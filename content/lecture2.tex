\section{Lecture 2: Geological methods for studying climate}

\subsection{Formanifera}

Single-celled organisms, members of Rhizarian protists, which are plenty
fossilised in the oceans.

\subsection{Oxygen isotope analysis}

$\delta^{18}$O$_{\text{water}}$, the 18-oxygen signature of water describes the
presence of the oxygen 18-isotope in a sample of water. The precise definition
is:

$
\delta^{18}\text{O}_{\text{water}} = 
\left[
\frac{
	\frac{{}^{18}O}{{}^{16}O}_{\text{sample}}
}{
	\frac{{}^{18}O}{{}^{16}O}_{\text{standard}}
}
-1
\right] \times 1000$‰

As per te standard value, we conventionally use:

$\frac{{}^{18}O}{{}^{16}O}_{\text{VSMOV}} = 0.002005$

VSMOW stands for Vienna Standard Mean Ocean Water

But it's very rare to have water from the past\footnote{
	One example is water trapped in bubbles in rocks.
}, so we have to compare with other elements like chemicals.
$\delta^{18}\text{O}_{\text{calcite}}$

$\delta^{18}\text{O}_{\text{VPDB}}
= 0.97 \delta^{18}\text{O}_{\text{VSMOW}} - 29.98 \text{‰}$

$\delta^{18}\text{O}_{\text{VSMOW}} = 1.03091 \times 
\delta^{18}\text{O}_{\text{VPDB}} + 30.91 \text{‰}$

VPDB = Vienna Pee Dee Belemnite, a specific fossil used as a standard, because
it is very homogenous.

We always compare oxygen isotopes in relation to a standard, not as absolute
values.

\subsection{Oxygen isotope fractionation}

Why is oxygen isotope a proxy?

In lower temperatures formanifera preferably in take the higher oxygen
isotope (${}^{18}$O) and in higher, the lighter isotope (${}^{16}$O).

As the shells of formanifera grow, they acquire the oxygen, and the oxygen
isotopes they intake depends on the surrounding temperature.

Shackelton and Kenneth (1975):
$T = 16.9 - 4.38[
\delta^{18}O_{\text{calcite}} - \delta^{18}O_{\text{water}}
]
+
0.10[
\delta^{18}O_{\text{calcite}} - \delta^{18}O_{\text{water}}
]^2
$

Given $^{18}O_{\text{calcite}}$ from a fossil we could calculate the
temperature from the past. But there is one problem, the value for water
is not constant on a geological timescale because of glaciers.

Water evaporates around the equator, it is then carried through clouds
northwards. Heavier isotope is more likely to fall down as rain, so the
further north the less of it is left in the clouds. Eventually the lighter
isotope gets trapped in ice (glaciers) and the heavier left in the ocean
waters.

Thus bigger values of $\delta^{18}O_{\text{water}}$ is correlated with
cooling or icehouse eras (presence of glaciers) or both.

The Earth has been cooling (with some minor interruptions) for the last 55 Myr.

In the last 1 Myr we can see a sawtooth pattern of interglaciar periods and
glaciar periods.

$\delta^{18}O$ from marine fossils over the past 500 Myr has been increasing
which suggests the climate has been cooling the whole time but we have other
proofs that it's not the case. That's example of proxy failing us.

\textbf{Detrending} is a pattern of deleting the high level trend for data
if we don't have an explanation for the trend. After detrending data may make
more sense even if it's "fake".

\subsection{Identifying cold periods in history}

\begin{itemize}
	\item ice-rafted debris
	\item glacial deposits
\end{itemize}

Oldest ocean floor on earth is 200 Myr old.

\subsection{Carbon isotopes}
$^{12}C$ and $^{13}C$ (don't confuse with the radioactive $^{14}C$). It's the
same principle as with oxygen isotopes, $^{13}C$ is heavier.

Hydrothermal (coming from volcanoes) $\delta^{13}C = -6$. Limestones form of
$CaCO_3$ or dolomite ($CaMg(CO_3)_2$). Thus limestones will have the same value
as volcanoes if nothing else changes it.

Now we add living organisms which preferentially take carbon 12 ($^{12}C$),
thus increasing $\delta^{13}C$ in the water. Extreme events happened around
2000 Myr ago and around 600 Myr ago -- very high and very low values of
$\delta^{13}C$ (blossom of life and death of life). Both are explained as
snowball events.

\subsection{Timescales}

4000 Myr ago -- 500 Myr ago: carbonates, $^{13}C$

500 Myr ago -- 100 Myr ago: marine fossils, $^{18}O$

65 Myr ago -- 0 Myr ago: foraminifera
