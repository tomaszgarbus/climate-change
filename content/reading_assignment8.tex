\section{Reading assignment: The Anthropocene: conceptual and historical
perspectives}
Will Steffen, Jacques Grinevald, Paul Crutzen and John McNeill

\subsection{Abstract}
\begin{itemize}
	\item the case for formally recognizing the Anthropocene as a new epoch
	\item advent of Industrial Revolution around 1800 is a logical start
	date for the epoch
\end{itemize}

\subsection{Introduction}
\begin{itemize}
	\item discovery of ozone hole over Antarctica with anthropogenic cause
	\item in addition to carbon cycle, humans are altering several other
	element cycles such as nitrogen, phosphorus and sulphur
	\item humans are also strongly modifying terrestrial water cycle
	\item driving the 6th major extinction event in Earth's history
	\item the term Anthropocene suggests that humans have become a 
	\textbf{geological} force of their own
\end{itemize}

\subsection{Antecendents of the Anthropocene concept}

\subsection{History of the human-environment relationship}
\begin{itemize}
	\item \textit{homo erectus} learned how to make stone tools,
	rudimentary weapons, and control fire
	\item shift from primarily vegetarian diet to omnivorous
	\item brain size grew 3fold
	\item development of spoken language
	\item written language, accumulation of knowledge
	\item China started burning coal in 11th century to support its iron
	industry
	\item coal started being used as fuel in England around 13th century
	\item London burned 360000 tonnes of coal annually by 1600s
	\item but China and England were still exceptions then
	\item pre-industrial events proposed as beginning of Anthropocene
	\begin{itemize}
		\item wave of extinctions of the Pleistocene\footnote{
		2.58 Ma to 11700 ago} megafauna, to which human hunting
		pressures played a role
		\item Neolithic Revolution in the early phases of the Holocene
		\footnote{11700 ago to now}. Two agriculture-related events:
		the clearing of forests about 8000 yrs ago and development of
		irrigated rice cultivation about 5000 yrs ago presumably
		emitted enough CO$_2$ to prevent the initiation of the next
		ice age
	\end{itemize}
\end{itemize}

\subsection{The beginning of the Anthropocene}
\begin{itemize}
	\item the Industrial Revolution had origins in Great Britain in the
	1700s
	\item end of agriculture as the most dominant human activity
	\item growing energy bottleneck: plants use less than 1 percent of the
	incoming solar radiation for photosynthesis and animals eating plants
	obtain only 10\% of energy from these plants
	\item discovery and exploitation of fossil fuels helped bypass that
	bottleneck
	\item Haber-Bosch process: energy intensive process synthesizing
	reactive nitrogen compounds from unreactive nitrogen in the atmosphere,
	creating fertilizer out of air
	\item between 1800 and 2000, human population grew from 1B to 6B,
	energy use grew 40x and economic production 50x
\end{itemize}

\subsection{The Great Acceleration}
\begin{itemize}
	\item the period from 1945 to 2000+
	\item population increased 3B to 6B
	\item economic activity grew 15x
	\item consumption of petroleum grew 3.5x
	\item number of motor vehicles rose from 40M to 700M by 1996
	\item over half of human population now lives in urban areas
	\item this 6th great extinction will be the 1st caused by a biological
	species
	\item atmospheric CO$_2$ concentration grew by 58ppm
\end{itemize}

\subsection{The anthropocene in the twenty-first century}
\begin{itemize}
	\item the Great Acceleration has become much more democratic and moved
	to developing countries, such as China, India, Brazil, South Africa,
	Indonesia
	\item developing countries have accounted for only about 20\% of total
	emissions since 1751 but contain about 80\% of world's population
	\item for 2004,  the emissions of developing countries grew to over
	40\% total
	\item \textbf{peak oil}
	\begin{itemize}
		\item maximum rate of the production of oil in any area under
		consideration, recognizing that it is a finite natural resource
		\item availability of oil beyond 2010?
		\item increased demand of about 2-3\% yr$^{-1}$ has been
		observed through 2000-2010
	\end{itemize}
	\item \textbf{phosphorus}
	\begin{itemize}
		\item the world may be close to peak phosphorus
		\item phosphorus along with nitrogen is a key element in
		fertilizers
		\item the demand for fertilizer will grow
	\end{itemize}
	\item in May 2010 scientists built a genome from its chemical
	constituents and used it to make synthetic life
	\item they created a bacterial chromosome, which was transferred into
	a bacterium where it replaced the original DNA, the bacteria cell then
	began replicating to produce a new set of proteins
	\item most widely discussed geo-engineering approach is artificially
	spreading aerosols into the stratosphere
	\item \textbf{planetary boundaries} -- approach explicitly based on
	returning the Earth's system to the Holocene domain
	\item the set of planetary boundaries defines the safe operating space
	for humanity with respect to the Earth system
\end{itemize}

\subsection{Societal implications of the Anthropocene concept}
