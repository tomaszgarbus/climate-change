%%%%%%%%%%%%%%%%%%%%%%%%%%%%%%%%%%%%%%%%%%%%%%%%%%%%%%%%%%%%%%%%%%%%%%%%%%%%%%%%

\section{Reading assignment 5: Snowball Earth by Paul F. Hoffman and Daniel
P. Schrag}

Between 750 Myr ago and 580 Myr ago 4 drastic climate reversals happened,
between icehouse and hothouse.

\textbf{Neoproterozoic} are spans from 1 Byr to 538.8 Myr ago.

Occurrence of glacial debris near sea level in the tropics.

Earth's landmasses were most likely clustered near the equator during the
global glaciations that took place around 600 million years ago.

Albedo: ice sheets reflect more of Sun's energy than dark seawater.

Budyko's climate model:
\textit{
But his climate simulations also revealed that this feedback can run out of
control. When ice formed at latitudes lower than around 30 degrees north or
south of the equator, the planet’s albedo began to rise at a faster rate
because direct sunlight was striking a larger surface area of ice per degree
of latitude. The feedback became so strong in his simulation that surface
temperatures plummeted and the entire planet froze over.
}

However, according to his model, the albedo feedback should have gotten out of
control and extinguished all life on Earth.

\textit{
The first of these objections began to fade in the late 1970s with the
discovery of remarkable communities of organisms living in places once
thought too harsh to harbor life. Seafloor hot springs support microbes that
thrive on chemicals rather than sunlight. The kind of volcanic activity that
feeds the hot springs would have continued unabated in a snowball earth.
Survival prospects seem even rosier for psychrophilic, or cold-loving,
organisms of the kind living today in the intensely cold and dry mountain
valleys of East Antarctica. Cyanobacteria and certain kinds of algae occupy
habitats such as snow, porous rock and the surfaces of dust particles encased
in floating ice.
}

\textit{
In 1992 Joseph L. Kirschvink, a geobiologist at the California Institute of
Technology, pointed out that during a global glaciation, an event he termed a
snowball earth, shifting tectonic plates would continue to build volcanoes and
to supply the atmosphere with carbon dioxide. At the same time, the liquid
water needed to erode rocks and bury the carbon would be trapped in ice.
With nowhere to go, carbon dioxide would collect to incredibly high levels,
high enough, Kirschvink proposed, to heat the planet and end the global freeze.
}

\textbf{4 stages of Snowball Earth}
\begin{itemize}
    \item \textbf{Snowball Earth Prologue}
    \begin{itemize}
    	\item landmass breakup 770 Myr ago
	\item formerly land-locked areas closer to oceanic sources of moisture
	\item increased rainfall erodes continental rocks
	\item global temperatures fall, carbon-dioxide gets removed from the
	air
	\item the ice sheets reflect sunlight, starting positive feedback
	\item planet covered in ice within a millenium
    \end{itemize}
    \item \textbf{Snowball Earth at its coldest}
    \begin{itemize}
    	\item avg global temperatures fall to $-50 \degree$
	\item oceans ice over to the depth of one kilometer
	\item most microscopic marine organisms die but some live around
	volcanic hot springs
	\item cold dry air, no precipitation
	\item due to no precipitation, volcanic carbon dioxide is not removed
	from the atmosphere
	\item the planet warms and sea ice slowly thins
    \end{itemize}
    \item \textbf{Snowball Earth as it thaws}
    \begin{itemize}
	\item concentrations of carbon dioxide in the atmosphere grow 1000x due
	to 10 Myr of normal volcanic activity
	\item greenhouse warming pushes temperatures to the melting point at
	the equator
	\item the open water absorbs more solar energy and speeds up the
	warming
    \end{itemize}
    \item \textbf{Hothouse aftermath}
    \begin{itemize}
	\item seawater evaporates and works with carbon dioxide to create even
	stronger greenhouse
	\item surface temperatures go to more than 50 degree Celsius
	\item carbonic acid rain errodes rock debris left in the wake of
	retreating glaciers
	\item swollen rivers wash bicarbonate and other ions into the oceans,
	where they form carbonate sediment
    \end{itemize}
\end{itemize}

\textbf{Animals}
\begin{itemize}
	\item all animals descended from the first eukaryotes, cells with a
	membrane-bound nucleus
	\item by the time of the first snowball episode more than 1 Byr later,
	eukaryotes had not developed beyond unicellular protozoa and
	filamentous algae
	\item the extreme climate might have pruned the eukaryote tree
	\item all 11 animal phyla ever to inhabit the Earth emerged within a
	narrow window of time in the aftermath of the last snowball event
\end{itemize}

%%%%%%%%%%%%%%%%%%%%%%%%%%%%%%%%%%%%%%%%%%%%%%%%%%%%%%%%%%%%%%%%%%%%%%%%%%%%%%%%
