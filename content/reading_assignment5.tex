%%%%%%%%%%%%%%%%%%%%%%%%%%%%%%%%%%%%%%%%%%%%%%%%%%%%%%%%%%%%%%%%%%%%%%%%%%%%%%%%

\section{Reading assignment 5: Snowball Earth by Paul F. Hoffman and Daniel
P. Schrag}

Between 750 Myr ago and 580 Myr ago 4 drastic climate reversals happened,
between icehouse and hothouse.

\textbf{Neoproterozoic} are spans from 1 Byr to 538.8 Myr ago.

Occurrence of glacial debris near sea level in the tropics.

Earth's landmasses were most likely clustered near the equator during the global
glaciations that took place around 600 million years ago.

Albedo: ice sheets reflect more of Sun's energy than dark seawater.

Budyko's climate model:
\textit{
But his climate simulations also revealed that this feedback can run out of
control. When ice formed at latitudes lower than around 30 degrees north or
south of the equator, the planet’s albedo began to rise at a faster rate
because direct sunlight was striking a larger surface area of ice per degree
of latitude. The feedback became so strong in his simulation that surface
temperatures plummeted and the entire planet froze over.
}

However, according to his model, the albedo feedback should have gotten out of
control and extinguished all life on Earth.

\textit{
The first of these objections began to fade in the late 1970s with the
discovery of remarkable communities of organisms living in places once
thought too harsh to harbor life. Seafloor hot springs support microbes that
thrive on chemicals rather than sunlight. The kind of volcanic activity that
feeds the hot springs would have continued unabated in a snowball earth.
Survival prospects seem even rosier for psychrophilic, or cold-loving,
organisms of the kind living today in the intensely cold and dry mountain
valleys of East Antarctica. Cyanobacteria and certain kinds of algae occupy
habitats such as snow, porous rock and the surfaces of dust particles encased
in floating ice.
}

\textit{
In 1992 Joseph L. Kirschvink, a geobiologist at the California Institute of
Technology, pointed out that during a global glaciation, an event he termed a
snowball earth, shifting tectonic plates would continue to build volcanoes and
to supply the atmosphere with carbon dioxide. At the same time, the liquid
water needed to erode rocks and bury the carbon would be trapped in ice.
With nowhere to go, carbon dioxide would collect to incredibly high levels,
high enough, Kirschvink proposed, to heat the planet and end the global freeze.
}

\textbf{4 stages of Snowball Earth}
\begin{itemize}
    \item \textbf{Snowball Earth Prologue}
    \item \textbf{Snowball Earth at its coldest}
    \item \textbf{Snowball Earth as it thaws}
    \item \textbf{Hothouse aftermath}
\end{itemize}

%%%%%%%%%%%%%%%%%%%%%%%%%%%%%%%%%%%%%%%%%%%%%%%%%%%%%%%%%%%%%%%%%%%%%%%%%%%%%%%%