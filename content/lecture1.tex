\section{Lecture 1: The controls of climate on geological timescales}

% \includegraphics[scale=0.2]{content/img/smiley.jpg}

\textbf{Time imbalance}: Coal takes hundreds of millions of years to accumulate
from fossils, but takes decades of burning to release. Accumulation happens on
\textbf{geological} timescale and release at \textbf{antropogenic} timescale.

Average Earth surface temperature is around $15 \degree$C.

\subsection{Climate factors}
Earth absorbs sunlight and radiates heat energy back into space.
These 3 factors control the process:
\begin{itemize}
	\item solar radiation
	\item albedo effect
	\item greenhouse effect
\end{itemize}

\subsubsection*{Solar radiation}

Some prerequisites for calculations:

\textbf{Stefan-Boltzmann law} describes the intensity of the thermal radiation
emitted by atter in terms of tat matter's temperature. Formula is
$E = \sigma T^4$, where
$\sigma = 5.670367 \times 10^{-8}W.m^{-2}.K^{-4}$

\textbf{Solar radiation} constant, in other words, the amount of energy emitted
by the Sun is $3.87 \times 10^{26}W$\footnote{
	When an object's velocity is held constant at one meter per second
	against a constant opposing force of one newton, the rate at which
	work is done is one watt: $1W = q kg \cdot m^2 \cdot s^{-3}$
}.

\textbf{Solar constant} $S_0$ describes the amount of energy received by a
given area one astronomical unit\footnote{roughly equal to average distance
Sun-Earth} away from the Sun. Let's calculate it:

$d_{Earth} = 149,597,870,700m$

Solar constant $S_0 = \frac{Q}{4\pi d^2} = 1362W.m^{-2}$. Since Earth is not
flat, but is a rotating sphere, this number is divided by 4, so the effective
energy received from Solar radiation is $342W.m^{-2}$.

Now from Stefan-Boltzmann's law, we can calculate the temperature:\\
$E = \sigma T^{4}$\\
$E = 342W.m^{-2}$\\
$T = (E.\sigma^{-1})^{1/4} = 6 \degree$\\

Now let's compare with values for Venus:\\
$d_{Venus} = 108 \times 10^9m$\\
$E_{Venus} = 658W.m^{-2}$\\
$T_{Venus} = -55 \degree$\\

\subsubsection*{Albedo}
Black seat: low albedo, white cat: high albedo

Venus has albedo effecto of $\alpha = 77\%$\\ 
Earth has albedo effecto of $\alpha = 30\%$

Of course, Earth's albedo is much harder to calculate because the terrain
varies a lot, compared to Venus which has a relatively uniform surface.

Venus radiates back to space $658W.m^{-2} \cdot 77\% = 504W.m^{-2}$.
Earth radiates back to space $342.m^{-2} \cdot 30\% = 103.m^{-2}$.

Taking into account albedo effect, Venus' surface temperature should be
$-46 \degree$ and Earth's $-18 \degree$.

\subsubsection*{Greenhouse effect}

Earth: greenhouse effect increases temperature by $32 \degree$.

Let's calculate how much the temperature increased due to greenhouse effect
since the preindustrial era, knowing that CO$_2$'s content in atmosphere
increased from 285ppm to 425ppm.

$\Delta T = 4.38 \ln \frac{{CO_2}_\text{present day}}
{{CO_2}_\text{preindustrial}} =
4.38 \ln \frac{425 \text{ppm}}{285 \text{ppm}} = 1.75 \degree$

\subsection{Earth's temperature summary}

$\underset{\text{Solar radiation}}{6 \degree} +
\underset{\text{Albedo}}{-24 \degree} +
\underset{\text{Greenhouse cases}}{32 \degree}$

\subsection{Faint Young Sun paradox}
We have fossils from 3.5 Byr ago. Earliest fossils are stromatolites\footnote{
Stromatolites are layered sedimentary formations created mainly by
photosynthetic microorganisms such as cyanobacteria, sulfate-reducing bacteria
and Pseudomonadota (formerly proteobacteria).
}.

Assuming the same percentage of $CO_2$ in the atmosphere, the average
temperature on Earth at that time (3.5 Byr ago) should have been around
$0 \degree$ (due to lower solar radiation), meaning no running water, which
precludes the possibility of life.

\subsection{Source of CO$_2$ on geological timescales}
Volcanoes

\subsection{Earth's thermostate -- chemical weathering}
Hydrolysis is the main mehcanism for removing CO$_2$ from the atmosphere.
Three key ingredients are minerals that make typical continental rocks, water
derived from rain, and CO$_2$ derived from the atmosphere.

The central equation for chemical weathering is:

$
\underset{\text{Silicate rock}}{CaSiO_3}
+
\underset{\text{Carbonic acid (soil)}}{H_2CO_3}
\rightarrow
\underset{\text{Shells of organisms}}{CaCO_3 + SiO_2 + H_2O}
$

\subsection{Chemical weathering of silicate rocks}
First stage of chemical weathering happens under the influence of rain:

$
\text{Granite} + H_2O + 2 CO_2 =
\text{Clay} + 2K^+ + 2 HCO_3^-
$

During the weathering, carbon dioxide switches from being a greenhouse gas
to being a sollute.

The bicarbonate ions are then carried by rivers and eventually end up in seas
and oceans. In the ocean, bicarbonates find calcium which they react with, and
make limestone, which is calcium carbonate.

$
2 HCO_3^- + Ca^{2+} = \underset{\text{limestone}}{CaCO_3} + H_2O + CO_2
$

We took 2 molecules of carbon from the atmosphere, and return only one, the
other one is deposited in limestone.
Thus the precipitation limestone is a sink.

Q: Can chemical weathering of silicate rocks compensate for anthropogenic
CO$_2$ emissions?

A: No, it is way too slow.
