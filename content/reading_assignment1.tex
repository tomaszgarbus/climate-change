\section{Reading assignment: Earth's Climate Chapter 4}

\subsection{Key terms}

\textbf{Greenouse era}: times when no ice sheets are present 

\textbf{Icehouse era}: tmes when ice sheets are present

\textbf{Faint young Sun paradox}: the mystery why the Earth's climate has
remained relatively stable throughout most of the planet's history, even
though the Sun shone 25\% to 30\% more faintly 4.55 Byr than today.

\textbf{Thermostat}: thermostat's role is to mitigate extreme temperature by
reacting to hot temperature with cooling down the system (e.g. house) and to
cold by heating up. We don't know what the Earth's thermostat was through the
history, recompensating for the faint young Sun. Candidates include chemical
weathering and life.

\textbf{Silicate materials}: examples include quartz and feldspar. Silicate
materials typically are made of positively charged cations (Na$^{+1}$,
K$^{+1}$, Fe$^{+2}$, Mg$^{+2}$, Al$^{+3}$ and Ca$^{+2}$) that are chemically
bonded to negatively charged SiO$_4$ (silicate) structures.

\textbf{Chemical weathering feedback}: chemical weathering creates a negative
feedback in the climate. Since chemical weathering is strongly correlated to
temperature and precipitation, we can distinguish two causal chains:

\begin{itemize}
  \item initial change $\rightarrow$ warmer climate $\rightarrow$
    increased temperature, precipitation, vegetation $\rightarrow$ increased
    chemical weathering $\rightarrow$ increased CO$_2$ removal by weathering
    $\rightarrow$ reduction of initial warming
  \item initial chage $\rightarrow$ colder climate $\rightarrow$ decreased
    temperature, precipitation, vegetation $\rightarrow$ decreased chemical
    weathering $\rightarrow$ decreased CO$_2$ removal by weathering
    $\rightarrow$ reduction of initial cooling 
\end{itemize}

\textbf{Gaia hypothesis}: in its weakest and commonly accepted form, it states
that as life-forms gradually developed in complexity, they played a
progressively greater role in chemical weathering and its control of Earth's
climate. In its most extreme version, it states that life evolved for the
purpose of regulating Earth's climate.

\textbf{Snowball Earth hypothesis}: the hypothesis that Earth was once nearly
frozen, around 715 to 640 million years ago. Climate scientists have found
evidence that glaciers existed on several continents during that time. Some
believe these continents were located in the tropics then, but its hard to
locate them back in time.

\subsection{Review questions}

\subsubsection{Why is Venus so much warmer than Earth today?}
Its atmosphere has 96\% CO$_2$ (compared to Earth's $0.2\%$), creating
a much stronger greenhouse effect, trapping much more heat.

\subsubsection{What factors explain why Earth is habitable today?}
Small greenhouse effect adding only $32 \degree$C to average temperature
in Earth's atmosphere.

\subsubsection{Why does the faint young Sun pose a paradox?}
Astrophysical models of the Sun's evolution indicate it was $25\%$ to $30\%$
weaker early in Earth's history. Climate model simulations show that the
weaker sun would have resulted in a completely frozen Earth for more than half
of its early history if the atmosphere had the same composition as it does
today.

Primitive life forms date back to at least 3.5 Byr ago, and their presence on
Earth is incompatible wit a completely frozen planet at that time.

\subsubsection{What evidence suggests that Earth has always had a long term
thermostat regulating its climate?}

The faint young Sun paradox, the specific evidence being prevalence of
water-deposited sedimentary rocks throughout Earth's early history.

\subsubsection{Why is volcanic input of CO$_2$ to Earth's atmosphere not a
candidate for its thermostat?}
Volcanic processes are diven by the heat sources located deep in the Earth's
interior and are well removed from contact with (and reactions to) climate
system.

\subsubsection{What climate factors affect the removal of CO$_2$ from the
atmosphere by chemical weathering?}
Temperature: weathering rates roughly doubl for each $10 \degree$C increase
in temperature.

Precipitation: increased rainfall boosts the level of groundwater held in
soils, and the water combines with CO$_2$ to form carbonic acid and enhance
the weathering process.

Vegetation: plants extract CO$_2$ from the atmosphere through photosynthesis,
and deliver it to soils, where it combines with groundwater to form carbonic
acid. It enhances the rate of chemical breakdown of minerals. Presence of
vegetation is estimated to increase the rate of chemical weathering by a factor
of 2 to 10.

\subsubsection{Where did the extra CO$_2$ from Earth's early atmosphere go?}
Sediments and rocks.

\subsubsection{What arguments support and oppose the Gaia hypothesis that
life is Earth's true thermostat?}
Critics say that too many of the active roles played by organisms in the
biosphere today are relatively recent developments in Earth's history. The also
point out that the very late appearance of shell-bearing oceanic organisms
near 540 million years ago means that life had played no obious role in
transferring the products of chemical weathering on land to the seafloor for
the preceeding 4 Byr.

Supporters claim that critics underestimate the role of primitve life-forms
such as algae in the ocan and microbes on land in Earth's earlier history.

Marine organisms that created oxygen through photosynthesis long ago are
believed to have enabled the development of oxygen-rich atmosphere 2.4 Byr.

