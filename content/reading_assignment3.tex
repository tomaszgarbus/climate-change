\section{Reading assignment 3: Astronomical control of solar radiation}:

\subsection{Key terms}

\textbf{Plane of the ecliptic}: the plane in which Earth moves around the Sun.

\textbf{Tilt}: The angle at which Earth is tilted away from the line
perpendicular to the plane of its orbit around the sun is $23.5 \degree$. The
tilt angle changes in cycles of 41000 years.

\textbf{Solstices}: The longest and shortest days of the year: June 21, Dec 21.

\textbf{Equinoxes}: The days in March and September when the lengths of night
and day become equal in each hemisphere.

\textbf{Perihelion}: "Close pass", when Earth is 153M km from the Sun. Happens
on January 3rd.

\textbf{Aphelion}: "Distant pass", when Earth is 158M km from the Sun. Happens
on July 4th.

\textbf{Wavelength}: Length of a cycle.

\textbf{Period}: The wavelength of a cycle, expressed in units.

\textbf{Frequency}: The inverse of the period of a cycle.

\textbf{Amplitude}: A measure of the amount by which cycles vary around their
long-term average.

\textbf{Modulation}: Behaviour in which the amplitude of peaks and valleys
changes in a repetitive or cyclic way.

\textbf{Sine waves}: Sinusoidals are perfect cycles because they are regular
both in period and in amplitude (isn't a sawtooth pattern too?).

\textbf{Eccentricity}: The measure of how elliptic (=not perfectly circular)
an ellipse is. It is defined as $\epsilon = \frac{\sqrt{a^2 - b^2}}{a}$.
Earth's orbit's eccentricity is not constant and has varied through time.

\textbf{Axial precession}: Earth's wobbling motion, caused by changing in the
direction of the tilt, at cycle length 25700 years.

\textbf{Precession of the ellipse}: A slow rotation of the entire orbit of the
Earth.

\textbf{Precession of the equinoxes}: The movement of equinoxes (and solstices)
around Earth's orbit which takes around 23000 years to complete. It results
from combined axial precession and precession of the ellipse.

\textbf{Precessional index}: An expression measuring impact of Earth's orbit's
eccentricity and the movement of equinoxes around the orbit. It is defined as
$\epsilon \sin \omega$, where $\epsilon$ is eccentricity and $\omega$ is the
current angle between the Earth-Sun lines at March 20 equinox and at
perihelion. Eccentricity \textit{modulates} the angular motion of the
precession of the equinoxes.

\textbf{Insolation}: Radiation arriving at the top of Earth's atmosphere.

\textbf{Caloric insolation seasons}: The summer caloric half-year is defined as
the 182 days when the incoming insolation exceeds the amount received during
the other 182 days.

\textbf{Time series analysis}: Techniques to extract rhythmic cycles embedded
in the records of climate.

\textbf{Spectral analysis}: One of time series analysis. Gradually sliding a
series of sine waves (with different phases and cycles) and if one has high
correlation with the plot, it is detected as component. Sounds like DFT.

\textbf{Power spectrum}: A type of plot where y-axis represents the amplitude
of the cycles, also known as power.

\textbf{Filtering}: A time series analysis technique, also known as bandpass
filtering.

\textbf{Aliasing}: A term that refers to false trends generated by
undersampling the true complexity in a signal.

\subsection{Review questions}:

\subsubsection{Why does Earth have seasons?}
Because of the tilt in relation to sun. It is summer when a given part of the
world is tilted towards and winter when in opposite direction. Therefore it is
summer in Australia when we have winter in Europe.

\subsubsection{When is Earth closest to the Sun in its present orbit? How does
this "close pass" position affect the amount of radiation received on Earth?}
In its present orbit, Earth is closest to the Sun on January 3. It causes the
winter radiation in the Northern Hemisphere and summer radiation in Southern
Hemisphere to be slightly stronger than they would be in a perfectly circular
orbit.

\subsubsection{Describe in your own words the concept of modulation of a cycle}
Modulation of a cycle are higher level patterns in its amplitude or loudness
if we talk about soundwave or intensity.

\subsubsection{Earth's tilt is slowly decreasing today. As it does so, are the
polar regions receiving more or less solar radiation in summer? In winter?}
As the tilt is decreasing, the Arctic and Antarctic cirles move closer to the
poles. That is, the area with polar day and polar night becomes slower. Polar
regions receive more radiation in the winter and less in the summer.

\subsubsection{How is axial precession different from precession of the
ellipse?}
Axial precession changes the direction of Earth's tilt, precession of the
ellipse moves the Earth's orbit.

\subsubsection{How does eccentricity combine with precession to control a key
aspect of the amount of insolation Earth receives?}
Precession affects how much solar radiation the Earth receives on the
solstices (the closer the solstice to the perihelion, the more radiation) in
a sinewave pattern. Eccentricity modulates this signal.
Since there are multiple cycles of eccentricity changes, with multiple lengths,
they nearly cancel each other out.

\subsubsection{Do insolation changes during summer and winter have the same or
opposite timing at any single location on Earth? Why or why not?}
They have have exactly opposite timing, they are exactly out of phase. This is
because the tilt that brings one pole closer to the Sun, also puts the other
pole farther from the Sun.

\subsubsection{Do the following changes occur at the same time (same year) in
Earth's orbital cycles?}

\textbf{Summer insolation maxima changes at both poles caused by changes in
tilt?}
No, they are out of phase.

\textbf{Summer insolation maxima in the tropics of both hemispheres by
precession?}
Yes, they are in phase.
