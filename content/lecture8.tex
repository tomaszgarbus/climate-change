\section{Lecture 8: A geological perspective on carbon sources and sinks}

\begin{itemize}
	\item IPCC: Intergovernmental Panel on Climate Change
	\item reservoir: a place where the carbon is stored
	\item for example, vegetation stores 450-650 Pg of carbon
	\item IPCC is using the word "stocks"
	\item 2009:
	\begin{itemize}
		\item 7.8 Pg -- amount of fossil fuels emitted to atmosphere
		per year
		\item 1.1 Pg -- from agriculture land use
	\end{itemize}
	\item 2019:
	\begin{itemize}
		\item 279 PgCyr$^{-1}$
	\end{itemize}
	\item we use C, not CO$_2$ so we account also for methane etc.
\end{itemize}

\subsection{Petagram}

1 petagram is $10^9$ tons.

\subsection{Carbon in the atmopshere (pre-industrial}

\begin{itemize}
	\item atmosphere 591 pgC
	\item slow cycle:
	\begin{itemize}
		\item volcanoes exchange 0.1PgC/yr with atmosphere
		\item mountains weather 0.1PgC/yr from atmosphere
	\end{itemize}
	\item fast cycle:
	\begin{itemize}
		\item ocean to atmosphere: 0.6 PgC/yr
		\item shallow waters to atmosphere: 1.5 PgC/yr
		\item soil from atmosphere: 2.1 PgC/yr
	\end{itemize}
	\item every 3rd molecule of carbon in the atmosphere was put there by
	humans
\end{itemize}

\subsection{Carbon reservoirs (stocks)}

\begin{itemize}
	\item atmosphere 591 PgC
	\item oceans 38700 PgC
	\item vegetation and soils 3350 PgC
	\item fossil fuels 928 PgC
\end{itemize}

\subsection{Natural climate solutions}
\begin{itemize}
	\item forest-related solutions (total capacity 4.4 PgC/yr):
	\begin{itemize}
		\item reforestation
		\item avoided forest fires
	\end{itemize}
	\item agriculture and grasslands (total capacity 1.3 PgC/yr):
	\begin{itemize}
		\item conservation agriculture (cover crops, no till, crop
		rotation)
		\item tree planting (instead of fences): capacity 0.3 PgC/yr
		\item biochar
	\end{itemize}
	\item wetlands (total capacity 0.7 PgC/yr):
	\begin{itemize}
		\item preservation of wetlands
	\end{itemize}
\end{itemize}

\subsection{Carbon Dioxide Removal (CDR)}
\begin{itemize}
	\item All analysed pathways limiting warming to $1.5 \degree$ with
	no or limited overshoot use CDR to some extent to neutralize emissions
	from sources for which no mitigation measures have been identified and,
	in most cases, also to achieve net negative emissions to return
	global warming to $1.5 \degree$ following a peak.
\end{itemize}
