\section{Reading assignment: The Paleocene-Eocene Thermal Maximum: A
Perturbation of Carbon Cycle, Climate, and Biosphere with Implications for the
Future}

\textbf{PETM}: Paleocene-Eocene Thermal Maximum, ca. 56 Myr ago

Period of carbon release below 20k years, the whole event lasted about 200k
years. The global temperature increase was $5-8 \degree$.

Kenneth and Stott 1991:
\begin{itemize}
	\item Ocean Drilling Program 690 off the coast of Antarctica
	\item Decline in oxygen isotope ratios indicating warming
	$3-4 \degree$ in surface water and $6 \degree$ in deep water
	\item Negative shift in $\delta^{13}C$ of benthic and planktic forams
	\item Rapid onset of the event, around 6k yrs
\end{itemize}

\textbf{CIE}: Carbon Isotopic Excursion

Dating CIE onset:

\begin{itemize}
	\item Using radiometric dates of marine ash layers and orbital tunings
	of marine sediments
	\item Dated around 56.011-56.293 Ma.
	\item Orbital timescales suggest total duration of 150-220 ka.
\end{itemize}

The PETM is defined by a global temperature rise that was initially inferred
from a $> 1$‰ negative excursion in in $\delta^{18}O$ of benthic foraminifera,
indicating a deep-water temperature increase of $5\degree$C.

A similar warming was inferred from the Mg/Ca ratios.

The absence of warming during PETM at polar latitudes implies lack of
ice-albedo feedback loop.

Indicators of massive carbon release at PETM:

\begin{itemize}
	\item large global negative CIE
	\item extensive dissolution of deep-ocean carbonates
\end{itemize}

\textit{The negative shift in carbon
isotope ($\delta^{13}$C) values shows that the carbon released was depleted in
$^{13}$C relative to the exogenic reservoir (ocean + atmosphere + biomass) and
was likely organic carbon because organisms discriminate against
$^{13}$C during biosynthesis.} The rapid onset (<20ka) indicates addition of
$^{13}$C-depleted carbon rather than reduction in organic carbon burial (100ka
timescale).

\subsection{Summary points}

1.The Paleocene-Eocene Thermal Maximum, which took place around 56 Mya and
lasted for around 200 ka, stands as the most dramatic geological confirmation
of the greenhouse theory -- increased CO$_2$ in the atmosphere warmed Earth's
surface.

2. The large release of organic $^{13}$C-depleted carbon caused a global
carbon isotopic excursion, widespread deep-ocean acidification, and carbonate
dissolution.

3. Carbon was removed from the exogenic pool on a timescale of 100 ka,
primarily through silicate weathering and eventual precipitation of carbonate
in the ocean and/or uptake by the biosphere and subsequent burial as organic
carbon.

4. Warming associated with the carbon release implies approximately two
doublings of atmospheric $p$CO$_2$ unless climate sensitivity was significantly
different during the Paleogene.

5. Although there was a major extinction of benthic foraminifera, most groups
of organisms did not suffer a mass extinction.

6. Geographic distributions of most kinds of organisms were radically
rearranged by $5-8 \degree$C of warming, with tropical forms moving poleward
in both maritime and terrestial realms.

7. Rapid morphological change occured in both maritime and terrestrial lineages
suggesting that organisms adjusted to climate change through evolution as well
as dispersal and local extirpation. Where best understood, these evolutionary
changes appear to be a response to nutrient and /or food limitation.

8. Research of the PETM and other intervals of rapid global change has been
driven by the idea that they provide geological parallels to future
anthropogenic warming, but much remains to be done to gain information that can
be acted on.

